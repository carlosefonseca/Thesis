\section{Architecture} % (fold)
\label{sec:architecture}

Eagle Eye is divided in two parts: The visualization and the backend. As their names point, the former is what the user will manipulate to view the images while the latter reads the images and generates the needed data for the visualization to work.

We will detail this to parts next but first we will go through some design decisions that have shaped the work. 


\subsection{Design Decisions} % (fold)
\label{sub:design_decisions}

\subsubsection{DeepZoom} % (fold)
\label{ssub:deepzoom}

After analyzing some visualization technologies that allowed easy display and manipulation of images\todo{Insert a bunch of useless tech}, Microsoft’s Silverlight, with its DeepZoom technology, proved to be the best choice.

DeepZoom enables the use of multiple resolution images for efficient display at various zoom levels and, for instance, is used for the display of gigapixel photos\todo{links}, where the user can view the whole image or zoom in the small details, or for collections of photos where the user can zoom between a view of multiple images and the details of a single image. So far, usages of DeepZoom have been restricted to promotion websites, art galleries and other closed usages.

\red{This should be about the multi-scale images and not about DeepZoom $\rightarrow$ }Our work aims at bringing this technology to the regular user, in a much simpler and dynamic way.

Although DeepZoom seemed a great technology, it relies on Silverlight, which by itself isn't as good as using a full-fledged desktop application framework like \ac{WPF} or other frameworks native to their platforms. This created some undesired limitations like the need to have two separated parts of the system, the visualization and the backend, and other smaller problems like limited access to the disk from the visualization part.

To create a DeepZoom application, some pre-processing is required before hand to create multiple versions of each image in various resolutions and also to create imagery of a global view of the collection, also in multiple resolutions. This enables DeepZoom to only load the appropriate set of images for a certain zoom state, keeping the bandwidth (if used on the Internet) and memory \red{requirements} to a manageable level. This required pre-processing   is done once on the backend. The visualization then uses the generated data to display the collection, being, at this point, totally independent from the original image files belonging to the user.

% subsubsection deepzoom (end)
% subsection design_decisions (end)




\subsection{Visualization} % (fold)
\label{sub:visualization}

As referred on the \ref{sub:design_decisions}.


% subsection visualization (end)





\subsection{Backend} % (fold)
\label{sub:backend}



% subsection backend (end)

% section architecture (end)


