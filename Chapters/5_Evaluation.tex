\chapter{Evaluation} % (fold)
\label{chapter:evaluation}

% 10/15 pages

To better understand how well Eagle Eye works, we did some user testing that will now detail.

\section{Objectives}

With our tests, we wanted to observe how various parts of Eagle Eye worked for a variety of people. 
We will try to answer the following questions, with our evaluation of the system:

\begin{itemize}
\item How the users react to the large quantity of images displayed simultaneously?
\item How easily do the users understand how to use the navigation controls and what could be improved?
\item Do users think the available disposition and sorting options are adequate and useful?
\item Can users understand what are the filters capable of, or there is a need for an improvement?
\item Can users extract interesting information from the collections by using Eagle Eye?
\end{itemize}

We will answer these questions with our evaluation.

\section{Testing Methodology}

Ideally, every test we made would be using the tested user's photo collection, which is much more interesting for them, but it is impractical due to the time required to pre-process the photos.

Therefore we chose to perform two different tests:
\begin{myitemize}
\item a \textbf{generic test} with a fixed collection allowing for faster mass testing;
\item \textbf{user specific tests} using their own photo collections. The users gave us their own photo collections for us to process and then perform the testing.
\end{myitemize}

The generic test allowed us to broadly analyze how users responded to the \ac{UI} and to have a sense of how they felt experienced it, without waiting for a collection. This allowed us to focus the user specific test on a smaller set of people and assess how they felt by using their 


 only a small set of people with  The user specific tests allowed us to  better assess how users reacted to the interaction with their own photos.

Testing was done by presenting the user with Eagle Eye's visualization interface only and ready to use. They never saw the backend or any of its processing steps, since it is currently less then polished and is not the most important part of this work. 

The interface and controls are then explained to the users and they are then allowed to play around and explore Eagle Eye. I was next to them, not only giving little hints and some occasional help, but also to provide some guidance and motif for their continued navigation and exploration on the system by asking to see some pieces of the collection. In the end, the users where asked to fill in a form describing their opinion of the system for statistical analysis.

Tests where performed on my personal laptop computer, connected to a 22" external monitor. The big screen is very important and helps provide a better experience while using Eagle Eye, for its capacity to display the images in a larger and more comfortable size.


\section{Characterization of users}

Most of the users tested were students in this institution, between the ages of 23 and 25 some  

\section{Results}

bloh

\section{Case Studies}

bluh

\section{Discussion}

blaaaaah


————————————————————

CRUZ

\begin{myitemize}

	\item * só usa "programas básicos"

	\item biblioteca com 833 fotos

	\item Foi demonstrada a interface…

	\item * rapidamente identificou padrões de cores

	\item só usou uma câmara. organização por device useless

	\item * acha giro/gosta bastante

	\item não usa keywords

	\item * algumas sugestões de melhorias

	\item tem fotos semelhantes mas que passam o tempo das stacks (não faz bursts nem braketing)

	\item * nunca teve a experiencia de mexer com tantas fotos ao mesmo tempo

	\item * consegue distinguir eventos

	\item ecra de 22" ~1600x1080

	\item * identifica a falta de slideshow e rápido fullscreen de fotos

	\item pedi todas as fotos que ele tirou no interrail à noite… ele não tinha fotos nenhumas…

	\item depois de algumas experiencias, escolheu as fotos escuras , ordenou por pasta e verificou que não tinha fotos à noite

	\item pedi fotos onde eu e ele aparecemos. procurou em photowalks onde fomos, escolheu grupos na organização por pasta e seleccionou imagens onde pareceu ter pessoas

	\item queixou-se de não conseguir navegar quando está nas selecções de fotos

	\item sugeriu usar o right click para seleccionar e left click para navegar


	\item mudou-se para a lib de teste com fotos minhas

	\item pedí para ver as fotos mais antigas, ordenou por data (linear) e seleccionou uma data de fotos no principio do canvas


	\item pedi quantas vezes é que eu fui tirar fotos ao parque das nações

	\item procurou tags com naçoes… queixou-se que não tinha o OR 

	\item outra hipotese era ir pelo path e escolher grupos

	\item meteu por data e viu


	\item muito preto


	\item qual é o evento onde tenho mais fotos escuras…

	\item procure pro path


	\item pede um botão de reset dos filtros


	\item com que camara tiro fotos mais escuras

	\item cor $\rightarrow$ device

\end{myitemize}



D4rch

\begin{myitemize}

	\item 2000 fotos

	\item gostou da interface e da interação


	\item pedi fotos tiradas fora do país

	\item pediu pelo OR

	\item path $\rightarrow$ seleccão de grupos



	\item parece que seleccionar images e aplicar um filto n funca



	\item pedi que fotos não eram dele

	\item Devices $\rightarrow$ grupos


	\item botao limpar filtros


	\item pedi para encontrar uma foto fixe, verde... como se tivesse que a por na parede

	\item procurou por cor e escolheu uma


	\item dá jeito para ver muita coisa

	\item diz que se percebe bem as fotos e identificou alguns conjuntos de fotos no seu tamanho reduzido


	\item --


	\item A aplicação demora um pouco a carregar nas fotos, 


	\item Pedi para encontrar uma foto de um golfinho, foi pelas fotos azuis e rapidamente encontrou uma foto de um golfinho a saltar


	\item Pedi para descobrir e mostrar quantas vezes fui tirar fotos ao parque das nações

	\item Ordenou por path e viu os nomes



	\item Pedi para procurar as primeiras fotos que tirei



	\item --


	\item bugs nos filtros

	\item or

	\item mais rápido

	\item considera que a disposição das imagens é mais ou menos decente.

	\item as stacks não são explicitas, não se percebe porque estão juntas

	\item não dá pra seleccionar uma imagem dentro de uma stack

	\item concorda com a falta de um slideshow

	\item gosta das cores do group names mas questiona se pode acontecer alguém confundir-se pois as a organização por cores tem as cores mas as outras não


	\item pedi para tentar identificar fotos em zoomout e identifica algumas, embora já conheça algumas. teve problemas a perceber fotos mais cinzentonas.

\end{myitemize}




