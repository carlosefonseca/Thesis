\chapter{Evaluation} % (fold)
\label{chapter:evaluation}

% 10/15 pages

To better understand how well Eagle Eye works, we did some user testing that will now detail.

\section{Objectives}

With our tests, we wanted to observe how various parts of Eagle Eye worked for a variety of people. 
We will try to answer the following questions, with our evaluation of the system:

\begin{itemize}
\item How the users react to the large quantity of images displayed simultaneously?
\item How easily do the users understand how to use the navigation controls and what could be improved?
\item Do users think the available disposition and sorting options are adequate and useful?
\item Can users understand what are the filters capable of, or there is a need for an improvement?
\item Can users extract interesting information from the collections by using Eagle Eye?
\end{itemize}

We will answer these questions with our evaluation.

\section{Testing Methodology}

Ideally, every test we made would be using the tested user's photo collection, which is much more interesting for them, but it is impractical due to the time required to pre-process the photos.

Therefore we chose to perform two different tests:
\begin{myitemize}
\item a \textbf{generic test} with a fixed collection allowing for faster mass testing;
\item \textbf{user specific tests} using their own photo collections. The users gave us their own photo collections for us to process and then perform the testing.
\end{myitemize}

The generic test allowed us to broadly analyze how users responded to the \ac{UI} and to have a sense of how they felt experienced it, without waiting for a collection. This allowed us to focus the user specific test on a smaller and more manageable set of people, and assess how they felt while interacting with the photos they know.

Testing was done by presenting the user with the Eagle Eye's visualization interface ready to use. They never saw the backend or any of its processing steps, since that part is currently less then polished, takes time to process, and is not what we are interested on testing.

The interface and controls are explained in detail to the users, from how to navigate the collection to the buttons on the \ac{UI}. They are then allowed to play around and explore Eagle Eye. I was always next to them, not only giving little hints and some occasional help, but also to provide some guidance and motif for their continued navigation and exploration of the system by asking to see some pieces of the collection. In the end, the users where asked to fill in a form describing their opinion of the system for statistical analysis.

Most tests where performed on my personal laptop computer, connected to a 22" external monitor. The big screen is very important and helps provide a better experience while using Eagle Eye, for its capacity to display the images in a larger and more comfortable size then my computer's 13" screen. Other tests had to be performed directly on the laptop, since I had to meet some users at their work locations on the university campus.

\section{Characterization of users}

Most of the users tested were students in this institution, between the ages of 22 and 25, from the various degrees of computer engineering. Some were avid photographers, others not so much.
\todo{…}


\section{Results}

The general feedback obtained is that users generally like the concept and the interaction of the system. They like to explore the library in different ways and play around with the dispositions. The majority have concluded that it needs some improvements in certain areas.

\subsection{\red{User behavior}}

Upon viewing how the system worked, users generally went ahead and played a while with the zooming, and viewed some images and groups at will while noticing how the groups can be differentiated when zoomed out. Later, users tried the different disposition options, generally in the order they are presented in the \ac{UI}. The motif for two date displays is usually questioned and explained and they move on to the color display which users generally like for the nice and clear color separation, although sometimes finding some images in odd locations. After viewing the differences in dispositions and view some images more zoomed in, the users tried the filtering or the image/group selection for reducing the amount of images displayed and focus only on part of the collection. The filter showed useful even though its bugs appeared once in a while. The manual selections were easier to use and for that captivated more usage from the users.


\subsection{System Usability Scale}

On the inquiry filled by our test users after the testing the system, we had a section with the \acf{SUS} questions. \ac{SUS}\cite{Brooke:1996ua} is a simple and generic ten question scale, giving a global view of the usability of a system. The \ac{SUS} result of inquiring \todo{20} people was \todo{75} points on a scale of 0-100 points, which is a fairly good mark for our system.

\begin{table}[h]
	\rowcolors{0}{light-gray}{white}
	\renewcommand{\arraystretch}{1.5}
	\centering
\begin{tabular}{l|c}
I think that I would like to \textbf{use this system frequently} & 3.4\\
I found the system \textbf{unnecessarily complex} & 2.0\\
I thought the system was \textbf{easy to use} & 3.9\\
I think that I would \textbf{need the support of a technical person} to be able to use this system & 1.3\\
I found the various functions in this system were \textbf{well integrated} & 3.5\\
I thought there was \textbf{too much inconsistency} in this system & 1.9\\
I would imagine that most people would \textbf{learn to use this system very quickly} & 3.8\\
I found the system very \textbf{cumbersome} to use & 2.3\\
I felt very \textbf{confident} using the system & 4.0\\
I needed to \textbf{learn a lot of things} before I could get going with this system & 1.5\\
\end{tabular}
\caption[Average of the results for each of the ten questions of \ac{SUS}.]{Average of the results for each of the ten questions of \ac{SUS}. Values range between 1.0 (strongly disagree with the question) and 5.0 (strongly agree). Gray rows are positive questions, where higher values are better, and white rows are negative questions, where lower values are better.}
\label{tab:sus}
\vspace{\baselineskip}
\end{table}

The averaged results to each of the answers of \ac{SUS} can be seen on Table \ref{tab:sus}. The questions' focus alternates between positive and negative characteristics for the system. Some important questions got good rates, like the ``easy to use'' and ``quick to learn'' with a rating of 4 of a maximum 5 and the ``unnecessarily complex'' question also getting the second best rate for a negative question, 2, which means users generally found the system to be easy to use and learn, although it still has some margin for improvement on the user experience area. Another important question is the ``I would like to use this system frequently'' which got a rating of 3.4, slightly lower from what we would like to see but still above the waterline. This reinforces the point above about improving the system but this time with features that make the user want to use the system more often.

\subsection{What users liked}

Questioned of what they liked on the system, and from what was seen from the interaction, we can confidently say users liked the general concepts of the system. I now present a summary of the users opinions of the good points of Eagle Eye:
\begin{itemize}
\item the photo display of a much greater amount of photos than what they've ever seen,  making visualization easier;
\item the easy zoom in and out to quickly see groups or photos in full size or in smaller size;
\item the possibility of still being able to identify groups of images which are small sized;
\item the various ways to quickly organize and access the images, specially the ability to view groups by events or colors;
\item the ability to selected images;
\item the smart use of space by grouping related images;
\end{itemize}

Essentially, people seem to like the new interaction methods to browse photos, specially if they know the photos themselves, which makes the overall usage of the system much more interesting.

\subsection{User suggested improvements}

During the course of the tests and on the questionnaire, some users suggested some improvements to our work. Some of these improvements were already in our mind but we didn't had enough time to implement them. We will now expose the suggestions we agree on:

\begin{itemize}
	\item Information to help the user use the system, e.g., help, guides or tutorial;
	\item Easier way to quickly view a single image/group in full size, like a double click;
	\item The filter should the improved, removing the bugs, giving some guidance to the user and allowing both \texttt{AND} and \texttt{OR} logics (currently only uses the former);
	\item The display should be organized hierarchically where appropriate, e.g., date, path and color based visualizations;
	\item The color measure should be improved;
	\item Event-based visualization, better then just dates or paths;
	\item Face detection should be improved to Person recognition;
	\item Improved method to select single images;
	\item Display the number of photos on screen;
	\item Better group separation.
\end{itemize}

Although we tried to make the user interface as clean and simple as possible, a couple users found some difficulties and suggested some simple guides. We think a simple introduction to the system is nice and helps the users, but it wouldn't make sense to spend time doing it at this point.


\subsection{Accomplished Objectives} 

We can now answer what objectives for Eagle Eye were accomplished. 

\subsubsection{How the users react to the large quantity of images displayed simultaneously?}
Users generally liked the large view and were surprised by how much can still be recognized, specially when working with their own photos and if they're grouped by events. 

\subsubsection{How easily do the users understand how to use the navigation controls and what could be improved?}
Users like the navigation controls but many felt it should be easier to go directly to a full view of one image and some agreed that keyboard navigation or slideshow-like controls should make the navigation easier.

\subsubsection{Do users think the available disposition and sorting options are adequate and useful?}
Users liked the ability to switch between dispositions and preferred some more the others.
Users generally liked the date display in groups but some would prefer that it would be more faithful to the time. Some suggested that it would be better if the tree map was organized with groups and subgroups so that all photos of a year/month got grouped together. Users understood the current need for the linear date display but found it hard to view anything and didn't spend much time with it.
The other preferred disposition was the color, the overall result of the grouping was pleasing to the eye of most users. The problem was when they started zooming in and some images revealed to be a mix of different colors from the color of the group they were in. This was a known problem of the method used. Another problem that revealed was the large size of groups like ``Gray'' or ``Black'' usually took great part of the screen, leaving all the more interesting colors to half of the screen. Some users even filtered out this black and grey groups by selecting and viewing only the colorful ones.
The path display wasn't very used since it ended up being much similar to the date display, except on some libraries where there folders weren't named by date but by event and made some sense to view the groups by these names. 

\subsubsection{Can users understand what are the filters capable of, or there is a need for an improvement?}
Some users understand what the filter is capable of when the autocomplete gives some suggestions, others could use some initial guidance to understand that a lot of things can be used on it. The biggest problem is that some of the information they would like to use can't be obtained automatically, like location (unless already on EXIF) or names of people, unless we hook into some third party app where the user have previously saved this data.

\subsubsection{Can users extract interesting information from the collections by using Eagle Eye?}


\section{Case Studies}

bluh

\section{Discussion}

blaaaaah


————————————————————

CRUZ

\begin{myitemize}

	\item * só usa "programas básicos"

	\item biblioteca com 833 fotos

	\item Foi demonstrada a interface…

	\item * rapidamente identificou padrões de cores

	\item só usou uma câmara. organização por device useless

	\item * acha giro/gosta bastante

	\item não usa keywords

	\item * algumas sugestões de melhorias

	\item tem fotos semelhantes mas que passam o tempo das stacks (não faz bursts nem braketing)

	\item * nunca teve a experiencia de mexer com tantas fotos ao mesmo tempo

	\item * consegue distinguir eventos

	\item ecra de 22" ~1600x1080

	\item * identifica a falta de slideshow e rápido fullscreen de fotos

	\item pedi todas as fotos que ele tirou no interrail à noite… ele não tinha fotos nenhumas…

	\item depois de algumas experiencias, escolheu as fotos escuras , ordenou por pasta e verificou que não tinha fotos à noite

	\item pedi fotos onde eu e ele aparecemos. procurou em photowalks onde fomos, escolheu grupos na organização por pasta e seleccionou imagens onde pareceu ter pessoas

	\item queixou-se de não conseguir navegar quando está nas selecções de fotos

	\item sugeriu usar o right click para seleccionar e left click para navegar


	\item mudou-se para a lib de teste com fotos minhas

	\item pedí para ver as fotos mais antigas, ordenou por data (linear) e seleccionou uma data de fotos no principio do canvas


	\item pedi quantas vezes é que eu fui tirar fotos ao parque das nações

	\item procurou tags com naçoes… queixou-se que não tinha o OR 

	\item outra hipotese era ir pelo path e escolher grupos

	\item meteu por data e viu


	\item muito preto


	\item qual é o evento onde tenho mais fotos escuras…

	\item procure pro path


	\item pede um botão de reset dos filtros


	\item com que camara tiro fotos mais escuras

	\item cor $\rightarrow$ device

\end{myitemize}



D4rch

\begin{myitemize}

	\item 2000 fotos

	\item gostou da interface e da interação


	\item pedi fotos tiradas fora do país

	\item pediu pelo OR

	\item path $\rightarrow$ seleccão de grupos



	\item parece que seleccionar images e aplicar um filto n funca



	\item pedi que fotos não eram dele

	\item Devices $\rightarrow$ grupos


	\item botao limpar filtros


	\item pedi para encontrar uma foto fixe, verde... como se tivesse que a por na parede

	\item procurou por cor e escolheu uma


	\item dá jeito para ver muita coisa

	\item diz que se percebe bem as fotos e identificou alguns conjuntos de fotos no seu tamanho reduzido


	\item --


	\item A aplicação demora um pouco a carregar nas fotos, 


	\item Pedi para encontrar uma foto de um golfinho, foi pelas fotos azuis e rapidamente encontrou uma foto de um golfinho a saltar


	\item Pedi para descobrir e mostrar quantas vezes fui tirar fotos ao parque das nações

	\item Ordenou por path e viu os nomes



	\item Pedi para procurar as primeiras fotos que tirei



	\item --


	\item bugs nos filtros

	\item or

	\item mais rápido

	\item considera que a disposição das imagens é mais ou menos decente.

	\item as stacks não são explicitas, não se percebe porque estão juntas

	\item não dá pra seleccionar uma imagem dentro de uma stack

	\item concorda com a falta de um slideshow

	\item gosta das cores do group names mas questiona se pode acontecer alguém confundir-se pois as a organização por cores tem as cores mas as outras não


	\item pedi para tentar identificar fotos em zoomout e identifica algumas, embora já conheça algumas. teve problemas a perceber fotos mais cinzentonas.

\end{myitemize}




