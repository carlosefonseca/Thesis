\chapter{Conclusions}
\label{chapter:conclusions}

% 3/4 pages
We will now go through what we think could be improved and conclude this work.



\section{Conclusion} % (fold)
\label{sec:conclusion}

In this work we have explored the current state of image browsers, their goods and bads. We have then defined multiple requirements for the work to be done and created an implementation of them.

Eagle Eye is … something something something…

We tested it and found the users generally liked the new way to view their images and found how interesting it is to identify such small images

It still has space for some improvements but its close to a point where it could be put on the market. 



conclusions

\red{old: In our view, with this additions, Eagle Eye would be a useful photo viewer that could be used by anyone that has some computer knowledge and a photo collection spread on the computer.}

% section concluding (end)




\section{Future Work} % (fold)
\label{future_work}

During the realization of this work, many ideas popped in our minds but we didn't had the time to work on them.

\subsection{Consolidation} % (fold)
\label{sec:consolidation}

The most important effort that must be done to this work is a rearrangement of the backend and its connection to the visualization.

As we've seen on chapter \ref{sub:backend}, both parts of this work are currently separated, with the backend being a command line utility. Our idea was to create an graphical application where the user could enter the folder paths he wants to add to Eagle Eye and then the application would do the needed work on the background, using free resources, and when it was done, the user could enter in the visualization mode without the need to open the internet browser. Better yet if the the system allowed things to be viewed while the processing was still running.

This interface would also provide some configuration options, for the system and its plugins.

% section consolidation (end)


\subsection{Feature Extraction Plugins} % (fold)
\label{sec:feature_extraction}

The second important thing to do is to improve the feature extraction plugins.

The color plugin, for instance, should be able to better identify colors in images instead of simply make a color average.

More data should also be extracted from the EXIF. As example, location data should be translated into place names\footnote{The action of turning GPS coordinates into addresses is called reverse geocoding. An example of a provider of such service: \url{http://code.google.com/apis/maps/documentation/geocoding/index.html\#ReverseGeocoding}}. Another plugin we see some benefit in creating would be a generator of keywords related to the capture dates. Currently we have the visualization reading the capture dates and generate keywords like ``Summer'' or ``May 2011'' for use as search tags. By transferring this to a feature extraction plugin, this data will be generated only once. It would then appear in the Suggestions of the Filter Bar as date options, allowing the gathering of sets of images from various years or seasons.

An additional plugin that would be interesting to develop would be some kind of hook into other application's metadata about photos. As an example, it's possible to access Picasa's face recognition and identification data about photos. By bringing this data into Eagle Eye, we could allow the user to perform searches for people's names without having to add those names as keywords.

% section feature_extraction (end)


\subsection{Visualization} % (fold)
\label{sec:visualization}

On the visualization part, the linear disposition, used mainly by the date view, should be improved to make more clear the distinction between days, months, years. A great idea that would be interesting to use would be the inclusion of something like the date bar in the work of Girgensohn et al. \cite{Girgensohn:2010} (\fig{girgensohn}). Another idea for a different disposition born from the stacking of similar photos discussed in section \ref{ss:stacks}, where one image is bigger than the rest, to help identify what is on what group.

We also think it needs to be more slideshow friendly. Slideshows are important part of any photo application, be it on the desktop or even on the web, since people always like to view their images. This work already displays images in fullscreen, it just needs a better interaction to repeatedly move to the next image, something that can be achieved easily with a button on the \ac{UI} and bindings to keyboard keys.


\subsubsection{Connect} % (fold)
\label{ssub:connect}

Another interesting point that would make the system stand out would be the connection between services like calendaring or location-tracking. Crossing their informations with our library could allow searching for photos of a certain event or location without the need for the library to contain that specific information. For instance, we could search for ``Mary's wedding'' on our application and, by looking into the calendar, find out what that day was and display the related photos.