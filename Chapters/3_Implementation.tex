\chapter{Eagle Eye}
\label{cha:eagle_eye}

% 30/40 pages

After having analysed the related work in the previous section, we will now detail our vision for the solution of the problem, followed by our implementation of that same solution.


\section{Requirements}
\label{section:requirements}

Our survey was based on various types of previous work from the last ten years. Image browsers and technology has evolved a lot from those years until today but there still isn't a definitive way for a user to look at its photo collection and understand its content and evolution. 

The vision we have is a system that takes the digital photographs residing in the user's computer, and display them all on the screen in various arrangements, revealing the patterns, similarities and differences between them.

For this to happen, the system must be able to handle tens of thousands of images and display them all on the screen while maintaining responsiveness. The system's \ac{UI} should be clear easy to use, allowing the user to navigate through the display of photos, by zooming and panning, and to to reorganise the photos in a number of ways.
The system should gather as much information as it can from the photos for instance, date and time, relevant colours, presence of people, type of photograph, location. While some of this information is already embedded in digital photographs as metadata, written by the digital camera when the photo was taken, others are usually not and need to be calculated or extracted. Faces and relevant colour information are an example of that and the system must be prepared to extract this features from the image. The system must allow other feature extraction methods to be easily added in the future. All this information will then be used by the user to reorganise the photos in display.


\section{Backend} % (fold)
\label{sec:backend}

\input{Chapters/Backend}

% section backend (end)


\section{Interface} % (fold)
\label{sec:interface}

% section interface (end)


\cleardoublepage