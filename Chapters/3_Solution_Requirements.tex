\chapter{Solution Requirements} % (fold)
\label{chapter:solution_requirements}

Our survey was based on various types of previous work from the last ten years. Image browsers and technology have evolved a lot since then but there still isn't a definitive way for a user to look at his larger photo collection and understand its content and evolution.



We have the vision of a system that displays a large set of user’s photos at the same time, in various arrangements, revealing patterns, differences and similarities between them.

We will now expose this vision by presenting the main goals we want to achieve with this thesis, as well as some of the guiding implementation requirements that we followed for a better end result.





\section{Main Goals} % (fold)
\label{s:main_goal}

The main goal of this thesis is to provide a different approach to the photo collection browsing methods.

More specifically, the work should:

\begin{itemize}
	\item{extract interesting information about the images — explained in detail in section\ref{s:features};}


	\item{provide an interaction with the full collection — we want to enable the users to easily view and interact with the entire collection at the same time, if they want to;}

	\item{efficiently display a large number of images in a single screen — this goal goes inline with the previous one: to be able to interact with the full collection, we should be able to display it all, at the same time, using techniques to make better use of available screen space;}

	\item{allow the manipulation of the display — we want the users to create different views that enable new perceptions of their collection, based on the extracted information.}

\end{itemize}

With this capabilities, our work should be able to provide the user not only with a better understanding of the collection but also with an easy and interesting way to visually combine and view photographs.


%be able to handle thousands of images and display them all on the screen while maintaining responsiveness and giving useful information. The system's \ac{UI} should be clear and easy to use, allowing the user to navigate through the display of photos, through zooming and panning, and to reorganize the display the photos in a number of ways.

%The system should also gather as much information as it can from the photos, such as date and time, relevant colors, presence of people, type of photograph, user organization or location. While some of this information is already embedded in today's digital photographs as metadata, written by the digital camera when the photo was taken, others are usually not and need to be calculated or extracted. Faces and relevant color information are an example of that and the system must be prepared to extract this features from the image. The system must provide some capability for other feature extraction methods to be easily added in the future. All this information will then be used by the user to reorganize and filter the photos on display.


%§ We will now detail the work done, taking this requirements into account.

% subsection main_goal (end)









\section{Implementation Requirements} % (fold)
\label{s:Implementation_Requirements}

In addition to the referred main goals, we set our selves some design goals, or requirements, for our implementation. With them, we want our work to get closer to a real application, that real users can use and have some flexibility for it to evolve with time.

Therefore, we set the following requirements:

\subsection{Ease of Use} % (fold)
\label{ssub:ease_of_use}

Ease of use is one of the most important characteristics of any piece of software and can shape how well it will sell. Much more attention has been given to the \ac{UX} in the last few years. Systems that are easier to use systems, get the job done faster, are more enjoyable to use and allow less experienced users to use it.

Since this is such an important characteristic, we aimed at providing a simple interaction from the beginning to the end, while also providing a powerful system, even though it could be even more refined in a few areas.

% subsubsection ease_of_use (end)



\subsection{Extensibility} % (fold)
\label{ssub:Extensibility}

The extensibility factor of a system is also important for the added value that can be obtained by quickly adding new features, either by the developers or by third parties. We made some parts of our work with this in mind, by allowing either external plugins or by generalization of code, allowing for future improvements with less trouble.

% subsubsection Extensibility (end)



\subsection{Performance} % (fold)
\label{ssub:Performance}

One of our main goals is to display a large set of images on the screen, but this brings problems since each image, in full-size, can take a good set of resources of the system. If we multiply this resources for a thousand images, we will not have a performent work.

Therefore, we set this requirement for having a performent work, that can be used with at least a few thousand images without taking down the system while using it.


% subsubsection Performance (end)

\red{What else?}

% subsection design_goal (end)




\section{Features of Images and Photographs}
\label{s:features}

%extract interesting information about the images

As referred before, we want our system to extract interesting information about the images and we will now explain this in detail.


Unlike textual data, images are a type of data where is not trivial to extract information from, in a computational environment. It's easy for us, common people, to understand what certain picture is showing. We can easily distinguish if there are, for instance, animals, people, flowers, buildings but it's hard for a computer to do the same, and it's even harder to understand if that certain photo of a building and a person was taken because of the building or the person or both.
\refs

There have been various developments in the feature extraction front \refs and is currently possible to perform various detections in images with various levels of satisfaction.

Some examples of working solutions for feature detection:

\begin{myitemize}
	\item{Face recognition — detection of the presence of faces in the images and their position \refs}
	\item{Object identification — identify objects in images \refs. The current approaches require large example databases of objects that take a lot of space, becoming hard to implement for users.}
	\item{Identification of perceptive colors — what are the main colors that users perceive in certain image \refs}
	\item{\todo{find more}}
\end{myitemize}

We want to use some of this methods to automatically extract information from the contents of images but, since this work is mostly focused on photography, there is another way to obtain really useful information that is EXIF Metadata.

EXIF Metadata is a standard format for metadata included in digitally captured media, like photographs captured by digital cameras. This devices save a lot of information on the image file, like the date and time of the capture, camera settings, camera orientation, location data and even detected faces\footnote{the availability of some of this data requires capable camera hardware and software that are getting more common nowadays.}. This is textual data and, therefore, much easier to retrieve and analyze than the content based methods we discussed above and is a great addition to them.

Mixing all this different data, users should be able to interact, filter, sort and organize their collection in various ways, obtaining new and different perspectives of their images.




% chapter solution_requirements (end)