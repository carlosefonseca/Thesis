\chapter{Introduction} % 5/6 pages
\label{chapter:introduction}


% \section{Prologue} % (fold)
% \label{sec:prologue}

Since the advent of digital formats, photography has become much more common among people, and photo collections have started to grow more than they used to. But storing digital photographs isn't that different from the past. In fact, people have been storing their photographs on folders on their computers instead of photo albums on the shelf. One could say computers can help people view, explore and find more photos in a shorter period of time, they certainly have the power for that but, in the end, they usually don't do much more than make the user look for the photo album (probably a folder or an ``album'' on some applications) and then flick through its pages (scrolling) until the photos the user was looking for appear.

This is true for the most common photo management software on the market, like Google's Picasa\footnote{\url{http://picasa.google.com}}, Apple's iPhoto\footnote{\url{http://www.apple.com/iphoto}} or Aperture \footnote{\url{http://www.apple.com/aperture}}, or Adobe's Lightroom \footnote{\url{http://www.adobe.com/lightroom}}. Although they bring some management improvements with them, the analogy above still applies. They all provide a way to select albums or folders and scroll through contents. They allow searching through some existing metadata, the only metadata that some of them generate is only face detection and end up having lots of buttons, toggles and options for their editing capabilities that clutter the interface.


\todo{talk more about photography and panoramas and HDR's and bursts and things! Photography, man!}


\section{Our vision} % (fold)
\label{ssub:our_vision}

% subsubsection our_vision (end)

We want to provide the users a totally different way to interact with their photos by focusing only on them. We want to provide the user a birds-eye view of his photos by showing everything and allowing him to drill down and view any image he wants at a larger size. Eagle Eye is not meant to be an editing platform but a better visualization tool.

With this work we will show that \red{this is a good alternative way to look at the users photo collection and that the user can learn more things from it}.


% section prologue (end)


\section{Contributions} % (fold)
\label{sec:contributions}

With our work, we have understood that visualization of large collections at the same time is not only feasible but also interesting to the users. 

We observed that images can be really small and still be identifiable by their owners, as long as they are integrated with others of the same event.

\todo{moar stuff here}

% section context (end)


\section{Structure of the Document} % (fold)
\label{ssub:structure_of_the_document}

We will start by going through some previous work related to ours (chapter \ref{chapter:related-work}) followed by an identification of the requirements for our work (chapter \ref{chapter:solution_requirements}). Next, we will see what was implemented and how it works (chapter \ref{cha:eagle_eye}) and how the users responded to it (chapter \ref{chapter:evaluation}). To conclude, we will discuss some work that could be done to improve this thesis (chapter \ref{future_work}) and final thoughts to conclude (chapter \ref{chapter:conclusions}).

% subsubsection structure_of_the_document (end)

