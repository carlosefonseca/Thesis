\chapter{Introduction} % 5/6 pages
\label{chapter:introduction}


\section{Prologue} % (fold)
\label{sec:prologue}

Since the advent of digital photography, people have been storing their photographs on folders on their computers instead of photo albums on the self. One could say computers can help people view, explore and find more photos in a shorter period of time, they certainly have the power for that but, in fact, they really don't do anything else then make the user look for the photo album (probably a folder or an ``album'' on some applications) and then flick through its pages (scrolling) until the photos that he was looking for appear.

This is true for the most common photo management software, like Google's Picasa\footnote{\url{http://picasa.google.com}}, Apple's iPhoto\footnote{\url{http://www.apple.com/iphoto}} or Aperture \footnote{\url{http://www.apple.com/aperture}} or Adobe's Lightroom \footnote{\url{http://www.adobe.com/lightroom}}. Although they bring some management improvements with them, the analogy above still applies. They all provide a way to select albums or folders and scroll through contents. They allow searching through some metadata and have lots of buttons and options for their editing capabilities.

We want to provide the user a totally different way to interact with his photos by focusing only on them. We want to provide the user a birds-eye view of his photos by showing everything and allowing him to drill down and view any image he wants at a larger size. Eagle Eye is not meant to be an editing platform but a better visualization tool.

With this work we will show that \red{this is a good alternative way to look at the users photo collection and that the user can learn more things from it}.


% section prologue (end)


\section{Context} % (fold)
\label{sec:context}


Present concepts...?
% section context (end)


\section{Motivation} % (fold)
\label{sec:motivation}

Alguns casos de uso
