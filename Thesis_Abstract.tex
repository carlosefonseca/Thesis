%%%%%%%%%%%%%%%%%%%%%%%%%%%%%%%%%%%%%%%%%%%%%%%%%%%%%%%%%%%%%%%%%%%%%%%%
%                                                                      %
%     File: Thesis_Abstract.tex                                        %
%     Tex Master: Thesis.tex                                           %
%                                                                      %
%     Author: Andre C. Marta                                           %
%     Last modified : 21 Jan 2011                                      %
%                                                                      %
%%%%%%%%%%%%%%%%%%%%%%%%%%%%%%%%%%%%%%%%%%%%%%%%%%%%%%%%%%%%%%%%%%%%%%%%

\chapter*{Abstract and keywords}

% Add entry in the table of contents as section
\addcontentsline{toc}{section}{Abstract and keywords}

Digital photography has been a part of people's lives for the past decade, growing on hard drives, but image visualization has not been evolving enough to accompany this growth. Most people keep photos on folders or use software that can't display a large number of photos at the same time, not providing a good overview of their collection.

This work grew from the features of other works of alternative image visualizations and tries to bring a better way for regular people to explore their photo libraries and learn more about them.

This work provides an extensible backend system intended for image and metadata processing, gathering information about the user's photos, currently colors, faces, dates, paths and keywords, but, since it's extensible, could be used for much more.

We also created a visualization application to use the processed data and provide it to the users, in an interface that is clean and simple. It starts by displaying all the added images on the screen, at the same time, in a grid, grouped by dates. From here, the user can zoom in and out and change the display to organize the images by colors, faces or paths, and can filter using a mix of any data gathered by the backend . This way, users can view and explore their libraries in a different way from what they have been doing.

\vfill

\textbf{Keywords:} photographs, image visualization, image browsing, feature extraction

\cleardoublepage

