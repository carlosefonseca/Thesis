%%%%%%%%%%%%%%%%%%%%%%%%%%%%%%%%%%%%%%%%%%%%%%%%%%%%%%%%%%%%%%%%%%%%%%%%
%                                                                      %
%     File: Thesis_Nomenclature.tex                                    %
%     Tex Master: Thesis.tex                                           %
%                                                                      %
%     Author: Andre C. Marta                                           %
%     Last modified : 21 Jan 2011                                      %
%                                                                      %
%%%%%%%%%%%%%%%%%%%%%%%%%%%%%%%%%%%%%%%%%%%%%%%%%%%%%%%%%%%%%%%%%%%%%%%%
%
% The definitions can be placed anywhere in the document body
% and their order is sorted by <symbol> automatically when
% calling makeindex in the makefile
%
% The \glossary command has the following syntax:
%
% \glossary{entry}
%
% The \nomenclature command has the following syntax:
%
% \nomenclature[<prefix>]{<symbol>}{<description>}
%
% where <prefix> is used for fine tuning the sort order,
% <symbol> is the symbol to be described, and <description> is
% the actual description.

% ----------------------------------------------------------------------

\chapter*{Acronyms}

% Add entry in the table of contents as section
\addcontentsline{toc}{section}{Acronyms}

\begin{acronym}[TDMA]

\acro{UI}[UI]{user interface}
\acro{SOM}[SOM]{self organising map}
\acro{CBIR}[CBIR]{content based image retrieval}
\acro{MP}[MP]{megapixel}
\acro{FEP}[FEP]{feature extraction plugin}
\acro{WPF}[WPF]{Windows Presentation Foundation}
\acro{EXIF}[EXIF]{Exchangeable image file format}

\end{acronym}
\vfill

\cleardoublepage