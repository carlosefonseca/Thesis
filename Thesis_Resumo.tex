%%%%%%%%%%%%%%%%%%%%%%%%%%%%%%%%%%%%%%%%%%%%%%%%%%%%%%%%%%%%%%%%%%%%%%%%
%                                                                      %
%     File: Thesis_Resumo.tex                                          %
%     Tex Master: Thesis.tex                                           %
%                                                                      %
%     Author: Andre C. Marta                                           %
%     Last modified : 21 Jan 2011                                      %
%                                                                      %
%%%%%%%%%%%%%%%%%%%%%%%%%%%%%%%%%%%%%%%%%%%%%%%%%%%%%%%%%%%%%%%%%%%%%%%%

\chapter*{Resumo e palavras chave}

% Add entry in the table of contents as section
\addcontentsline{toc}{section}{Resumo e palavras chave}

A fotografia digital tem feito parte da vida das pessoas na ultima década, crescendo nos discos rígidos, mas a visualização de imagens não tem vindo a evoluir o suficiente para acompanhar este crescimento. A maior parte das pessoas guarda as fotos em pastas ou utiliza programas que não conseguem mostrar grandes quantidades de fotografias ao mesmo tempo, não fornecendo uma visão global da biblioteca.

Este trabalho cresceu das características de outros trabalhos de visualizações alternativas de imagens e tenta trazer a utilizadores comuns uma melhor maneira de explorar as suas bibliotecas de fotografias e aprender mais sobre elas.

Este trabalho fornece um sistema extensível para a extracção e agregação de características das imagens do utilizador e respectivos meta-dados, Actualmente são extraidas cores, rostos, datas, pastas e palavras-chave, mas, já que é extensível, poderá ser utilizado para muito mais.

Também criámos uma aplicação de visualização que usa os dados processados e os disponibiliza aos utilizadores numa interface limpa e simples. A aplicação começa por apresentar no ecrã, numa grelha, todas as imagens adicionadas ao programa, agrupados por datas. A partir daqui, o utilizador pode alterar a visualização ampliando-a ou reduzindo-a, organizando as imagens por cores, rostos ou pastas, podendo ainda filtrar utilizando uma mistura de todos os dados recolhidos pelo sistema de processamento. Desta forma, os utilizadores podem visualizar e explorar suas bibliotecas de uma forma diferente daquilo a que estão habituados.

\vfill

\textbf{Palavras-chave:} fotografias, visualização de imagens, exploração de imagens, extração de caracteristicas

\cleardoublepage

