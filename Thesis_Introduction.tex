%%%%%%%%%%%%%%%%%%%%%%%%%%%%%%%%%%%%%%%%%%%%%%%%%%%%%%%%%%%%%%%%%%%%%%%%
%                                                                      %
%     File: Thesis_Introduction.tex                                    %
%     Tex Master: Thesis.tex                                           %
%                                                                      %
%     Author: Andre C. Marta                                           %
%     Last modified : 21 Jan 2011                                      %
%                                                                      %
%%%%%%%%%%%%%%%%%%%%%%%%%%%%%%%%%%%%%%%%%%%%%%%%%%%%%%%%%%%%%%%%%%%%%%%%

\chapter{Stuff}
\label{chapter:introduction}

Insert your chapter material here...

% ----------------------------------------------------------------------
\section{Motivation}
\label{section:motivation}

Relevance of the subject...


% ----------------------------------------------------------------------
\section{State-of-the-art}
\label{section:state}

Insert your section material with the appropriate citations.
These can be cited in the following way: \\

Citation mode \#1 - \quad \cite{jameson:adjointns}

Citation mode \#2 - \quad \citet{jameson:adjointns}

Citation mode \#3 - \quad \citep{jameson:adjointns}


Citation mode \#4 - \quad \citet*{jameson:adjointns}

Citation mode \#5 - \quad \citep*{jameson:adjointns}


Citation mode \#6 - \quad \citealt{jameson:adjointns}

Citation mode \#7 - \quad \citealp{jameson:adjointns}


Citation mode \#8 - \quad \citeauthor{jameson:adjointns}

Citation mode \#9 - \quad \citeyear{jameson:adjointns}

Citation mode \#10 - \quad \citeyearpar{jameson:adjointns}


\subsection{Tables}
\label{subsection:tables}

Insert your subsection material and for instance a few tables...

\begin{table}[h!]
  \begin{center}
    \begin{tabular}{|c|c|}
      \hline
      item 1 & item 2 \\
      \hline
      item 3 & item 4 \\
      \hline
    \end{tabular}
  \end{center}
  \caption[Table caption shown in TOC]{Table caption}
  \label{table:simple}
\end{table}

Make reference to Table \ref{table:simple}.

\begin{table}[!htb]
  \begin{center}
    \begin{tabular}{lccc}
      Model           & $C_L$ & $C_D$ & $C_{M y}$ \\
      \hline
      Euler           & 0.083 & 0.021 & -0.110    \\
      Navier--Stokes  & 0.078 & 0.023 & -0.101    \\
      \hline
    \end{tabular}
  \end{center}
  \caption{Aerodynamic coefficients.}
  \label{tab:aeroCoeff}
\end{table}

Here is an example of a table with merging columns:

\begin{table}[!htb]
  \begin{center}
    \begin{tabular}[]{lrr}
      \hline
                     & \multicolumn{2}{c}{\underline{Virtual memory [MB]}} \\
                     & Euler       & Navier--Stokes \\
      \hline
      Wing only      &  1,000      &    2,000       \\
      Aircraft       &  5,000      &   10,000       \\
      (ratio)        & $5.0\times$ & $5.0\times$    \\
      \hline
    \end{tabular}
  \end{center}
  \caption{Memory usage comparison (in MB).}
  \label{tab:memory}
\end{table}


\subsection{Drawings}
\label{subsection:drawings}

Insert your subsection material and for instance a few drawings...

The schematic illustrated in Fig.~\ref{fig:algorithm} can represent some sort of algorithm.

\begin{figure}[!htb]
  \centering
  \scriptsize
%  \footnotesize 
%  \small
  \setlength{\unitlength}{0.9cm}
  \begin{picture}(8.5,6)
    \linethickness{0.3mm}

    \put(3,6){\vector(0,-1){1}}
    \put(3.5,5.4){$\bf \alpha$}
    \put(3,4.5){\oval(6,1){}}
    %\put(0,4){\framebox(6,1){}}
    \put(0.3,4.4){Grid Generation: \quad ${\bf x} = {\bf x}\left({\bf \alpha}\right)$}

    \put(3,4){\vector(0,-1){1}}
    \put(3.5,3.4){$\bf x$}
    \put(3,2.5){\oval(6,1){}}
    %\put(0,2){\framebox(6,1){}}
    \put(0.3,2.4){Flow Solver: \quad ${\cal R}\left({\bf x},{\bf q}\left({\bf x}\right)\right) = 0$}

    \put(6.0,2.5){\vector(1,0){1}}
    \put(6.4,3){$Y_1$}

    \put(3,2){\vector(0,-1){1}}
    \put(3.5,1.4){$\bf q$}
    \put(3,0.5){\oval(6,1){}}
    %\put(0,0){\framebox(6,1){}}
    \put(0.3,0.4){Structural Solver: \quad ${\cal M}\left({\bf x},{\bf q}\left({\bf x}\right)\right) = 0$}

    \put(6.0,0.5){\vector(1,0){1}}
    \put(6.4,1){$Y_2$}

    %\put(7.8,2.5){\oval(1.6,5){}}
    \put(7.0,0){\framebox(1.6,5){}}
    \put(7.1,2.5){Optimizer}
    \put(7.8,5){\line(0,1){1}}
    \put(7.8,6){\line(-1,0){4.8}}
  \end{picture}
  \caption{Schematic of some algorithm.}
  \label{fig:algorithm}
\end{figure}

\cleardoublepage

